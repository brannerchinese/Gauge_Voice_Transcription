\documentclass{beamer}

\usetheme{Antibes}
%\usepackage{beamerthemesplit} % Activate for custom appearance
%\usecolortheme{sidebartab}
\usecolortheme{wolverine}
%\usecolortheme{spruce}
\setbeamertemplate{footline}[frame number]

\usepackage{amsmath}

\usepackage[hang,multiple]{footmisc}		% for better footnotes

\usepackage{lipsum}

\graphicspath{{./graphics/}}
\DeclareGraphicsExtensions{.jpg,.jpeg,.png}
\newcommand\gr[2]{\scalebox{#2}{\includegraphics{#1}}}
% the following scales large graphics down to textwidth or textheight
\usepackage{calc}
\newlength{\imgwidth}
\newcommand\grw[1]{%   
    \settowidth{\imgwidth}{\includegraphics{#1}}%
    \setlength{\imgwidth}{\minof{\imgwidth}{\textwidth}}%
    \includegraphics[width=\imgwidth]{#1}%
}
\newlength{\imgheight}
\newcommand\grh[1]{%   
    \settoheight{\imgheight}{\includegraphics{#1}}%
    \setlength{\imgheight}{\minof{\imgheight}{\textheight}}%
    \includegraphics[height=\imgheight]{#1}%
}

\usepackage{media9}			% for sound files
\addmediapath{{MP3s/}}

\usepackage{url}			% allows URLs to be formatted specially
\usepackage{ulem}			% allows underlining

\usepackage{hyperref}		% allows links to be placed in text
\definecolor{links}{HTML}{999999}
\hypersetup{colorlinks,linkcolor=links,urlcolor=links}

\usepackage{alltt}

\usepackage{listings}
\lstset{language={[LaTeX]TeX},basicstyle=\ttfamily\footnotesize,frame=lines,captionpos=b,tabsize=4,keywordstyle=\color{blue},commentstyle=\color{dark-green},stringstyle=\color{red},numbers=left,numberstyle=\tiny,numbersep=5pt,breaklines=true,showstringspaces=false,extendedchars=true}

\title{Interfering with Voice-Recognition Software}
\author{David Branner}
\date{Hack and Tell, New York City\\ 4 February, 2014}



\usepackage{showexpl} 

\lstloadlanguages{[LaTeX]Tex} 
\lstset{% 
     basicstyle=\ttfamily\small, 
     commentstyle=\itshape\ttfamily\small, 
     showspaces=false, 
     showstringspaces=false, 
     breaklines=true, 
     breakautoindent=true, 
     captionpos=t 
} 

\begin{document} 

\maketitle

\section[Outline]{}
\frame{\footnotesize\tableofcontents[subsectionstyle=hide]}


\section{Introduction}
\subsection{What \LaTeX{} is}
\frame{\frametitle{What \LaTeX{} is}
\LaTeX{} is a high-level implementation of \TeX{}.

\vskip12pt \uncover<2->{\TeX{} is basically a math typesetting-system. It has been generalized to be useful in application to a variety of typographic and word-processing tasks.}

\vskip12pt \uncover<3->{Beyond that, it is also superbly useful for automatically generating high-quality PDFs from the output of a computer program.}
}

\frame{ 
\includemedia[
  addresource=MP3s/gauge_voice_transcription_01_natural.mp3,
  flashvars={
    source=MP3s/gauge_voice_transcription_01_natural.mp3
   &autoPlay=true
  }
]{\fbox{Play}}{APlayer.swf}
}

\frame{ 
Important ideas:\begin{enumerate}
\item<2->  \textbf{symbols}: \texttt{abcDEF}, \texttt{123}, \texttt{.;!},  etc.
\item<3->  \textbf{values}: \texttt{10, 10pt}, used in settings and calculation
\item<4-> \textbf{commands}\uncover<2->{: \texttt{\textbackslash somecommand[options]\{argument\}}}
\item<5-> \textbf{environments}\uncover<6->{: \begin{alltt}
\textbackslash begin\{someenvironment\}

...

\textbackslash end\{someenvironment\}
\end{alltt}}
\end{enumerate}
}

\section{Math typesetting}

\frame{\frametitle{Math typesetting}
Math typesetting is the first skill you should learn, since there are tools to make this easy.
}

\frame{
\begin{table}[htdp]
\caption{Important types of commands and operators in math}
\begin{center}\footnotesize
\begin{tabular}{|ll|ll|}\hline
  code & meaning & example or comparison & output \\\hline
  \texttt{\textasciicircum} & superscript & \texttt{a\textasciicircum \{b+c\}}  & \(a^{b+c}\) \\\hline
  \texttt{\_} & subscript & \texttt{a\_\{b+c\}} & \(a_{b+c}\) \\\hline
  \uncover<2->{\texttt{\textasciitilde} & ``nobreak space'' & \texttt{a\textasciitilde b} vs. \texttt{a b} & \(a~b\) vs. \(a b\) \\\hline
  \uncover<3->{\texttt{\textbackslash delta} & \(\delta\) & \texttt{\textbackslash Delta} & \(\Delta\) \\\hline
  \uncover<4->{\texttt{\textbackslash sin x} & \(\sin x\) & cf. \texttt{sin x} & \(sin x\) \\\hline
  \uncover<5->{\texttt{\textbackslash ldots} & \(\ldots\) & cf. \texttt{...\textasciitilde \textbackslash cdots} & ... \(\cdots\) \\\hline
  \uncover<6->{\texttt{\textbackslash int} & \(\int\) & \texttt{\textbackslash int\_a\textasciicircum b} & \(\int_a^{b}\) \\\hline
  \texttt{\textbackslash sum} & \(\sum\) & \texttt{\textbackslash sum\_\{n=0\}\textasciicircum \textbackslash infty} & \(\sum_{n=0}^\infty\)   \\\hline
  \uncover<7->{\texttt{\textbackslash not} & negation strike-through & \texttt{\textbackslash in, \textbackslash not\textbackslash in} & \(\in, \not\in\) \\\hline
  \uncover<8->{\texttt{\textbackslash vec\{\}} & vector diacritic & \texttt{\textbackslash vec\{a\}} & \(\vec{a}\) \\\hline
  \uncover<9->{\texttt{\textbackslash mathbb\{\}} & math ``blackboard'' font & \texttt{\textbackslash mathbb\{R\}} & \(\mathbb{R}\) \\\hline
  \uncover<10->{\texttt{\textbackslash frac\{\}\{\}} & fraction & \texttt{\textbackslash\{x\}\{y\}} & \(\frac{x}{y}\) \\\hline}}}}}}}}}
\end{tabular}
\end{center}
\label{default}
\end{table}%
\vskip-24pt See slide \#\ref{CodeCogs} for information about hands-on practice.
}

\subsection{``Inline'' vs. ``display'' math environments}
\frame[containsverbatim]{\frametitle{``Inline'' vs. ``display'' math environments} 
\begin{LTXexample} 
Inline: \(x^n\)
\vskip12pt
Display: \[x^n\]
\end{LTXexample} } 

\frame[containsverbatim]{ 
\begin{LTXexample} 
Inline: \(x_n\)
\vskip12pt 
Display: \[x_n\]
\end{LTXexample} } 

\frame[containsverbatim]{ 
\begin{LTXexample} 
\[\lim_{n\rightarrow\infty}
 \frac{1}{n}=0\]
\end{LTXexample} } 

\frame[containsverbatim]{ 
\begin{LTXexample} 
\[\sum_{n=1}^{\infty}
 \frac{1}{n^2}=
 \frac{\pi^2}{6}\]
\end{LTXexample} } 

\frame[containsverbatim]{ 
\begin{LTXexample} 
\[\int_a^b x\ dx = 
 \frac{b^2-a^2}{2}\]
\end{LTXexample} }

\section{Some tools of use in documenting code}



\subsection{Tables and graphics}
\frame{\frametitle{Tables and graphics}
We'll use a real-time example of \texttt{table} or \texttt{tabular}. (leaving the slides)
}

\subsection{Aligned formulas}
\frame{\frametitle{Aligned formulas}
We'll use a real-time example of \texttt{align}. (leaving the slides)
}

\section{Styling non-math text}

\subsection{Basic text styling}
\frame[containsverbatim]{\frametitle{Basic text styling}
\begin{LTXexample} 
Basic styling of text for \textbf{boldface}, \textit{italics}, \textsc{small caps}, \uline{underlining}, \sout{strikethrough} (the last two using the \texttt{ulem} package), etc.
\end{LTXexample}
There are countless others. Notice that these operate transparently via a kind of markup.
}

\subsection{Footnotes and cross-references}
\frame{\frametitle{Footnotes and cross-references}
Footnotes are placed into text using \texttt{\textbackslash footnote\{\}}:

\uncover<2->{\vskip12pt A label can be placed in the text as \texttt{\textbackslash label\{someLabelName\}} and then referred to as \texttt{\textbackslash ref\{someLabelName\}}. You'll see some examples below, on slide \#\ref{firstUseOfRef}.}

\uncover<3->{\vskip12pt Considerably more elaborate behaviors are available through specialized packages.}
}

\subsection{Text filler}
\frame[containsverbatim]{\frametitle{Text filler}\scriptsize 
Here is paragraph number six of the standard text-fill.
\begin{LTXexample} 
\usepackage{lipsum}
\lipsum[6]
\end{LTXexample}
}

\subsection{Inserting graphics}
\frame[containsverbatim]{\frametitle{Inserting graphics}
The following code inserts a graphic up to the maximum width of the text-area available.
\begin{lstlisting}
\usepackage{calc}     % allows calculations
\newlength{\imgwidth} % declares `variable'
\newcommand\grw[1]{   % declares `function'
 \settowidth{\imgwidth}{\includegraphics{#1}}%
 \setlength{\imgwidth}
   {\minof{\imgwidth}{\textwidth}}%
 \includegraphics[width=\imgwidth]{#1}%
}
\end{lstlisting}
}


\subsection{Getting help}
\frame{\frametitle{Getting help}
The slide above is probably a little frightening. 

\uncover<2-> {\vskip12pt The number of packages and lower-level commands you are likely to need is not actually very large for most purposes. But it is daunting for most people at first, so I recommend using a manual (Mittelbach \textit{et al.} is the best --- see slide \#\ref{mittelbach}).}

\uncover<3-> {\vskip12pt I also recommend using on-line resources (discussed on slide \ref{bestHelp}).\label{firstUseOfRef}}

\uncover<4-> {\vskip12pt The command above appears in the preamble. Let me illustrate what the document body of a real-world example looks like. (leaving the slides)}
}


\section{Slides}

\frame{\frametitle{Slides}
We'll use a real-time example of \texttt{beamer}. (leaving the slides)
}


\section{Automating the production of PDFs as program output}

\frame{\frametitle{Automating the production of PDFs as program output}
We'll use a real-time example from a program of mine. (leaving the slides)

\uncover<2-> {\vskip12pt Briefly, we place the preamble and any static pieces of \LaTeX{} code into files whose contents can be read in, and then we either use templating or other string-services to typeset the data.}
}

\section{Turing-completeness and its consequences}

\frame{\frametitle{Turing-completeness and its consequences}
\TeX{} is a Turing-complete language.}

\frame{
The list seems endless\dots\begin{enumerate}
\item \href{http://tex.stackexchange.com/questions/29402/}{How do I make my document look like it was written by a Cthulhu-worshipping madman?} --- especially the graphic below the answer \href{http://tex.stackexchange.com/a/29458/3935}{here}
\item \href{http://tex.stackexchange.com/questions/62570/}{Letterpress effect through PSTricks or Tikz}
\item \href{http://www.texample.net/tikz/examples/feature/overlays/}{TikZ and PGF examples}
\item \href{http://tex.stackexchange.com/a/104288/3935}{One answer to ``What is the most bizarre thing you have seen done with TeX?''}
\item \href{http://tex.stackexchange.com/questions/91678/}{Self-replication}
\item \href{http://tex.stackexchange.com/questions/88751}{Text spirals}
\end{enumerate}
See \url{http://tex.stackexchange.com/questions/tagged/fun}.}


\section{References}

\frame[containsverbatim]{
To practice math syntax, or for graphical self-help, go to \href{http://codecogs.com/latex/about.php}{CodeCogs site} and choose the ``standalone editor.''\label{CodeCogs}

\vskip12pt Applications of \LaTeX{} syntax:
\begin{enumerate}
\item \href{http://www.mathjax.org/}{MathJax}, for use in HTML

\item \href{http://www.mediawiki.org/wiki/Manual:Math}{Description of use in MediaWiki}

\item \href{http://www.wolframalpha.com/}{Go to WolframAlpha} and enter 
\begin{verbatim}
\int_0^\infty\frac{x}{e^x}dx
\end{verbatim}
and then hit return. It should display (and then solve) \[\int_0^\infty\frac{x}{e^x} dx\]

\end{enumerate}
}

\frame{
Best help sites:
\begin{enumerate}
\item<2-> \href{http://tex.stackexchange.com/about}{\TeX{}-\LaTeX{} Stack Exchange} (A superbly supportive forum environment. You'll find many other questions answered capably elsewhere on the stackoverflow.com site, too.)
\item<3-> \href{http://www.tug.org}{\TeX{} Users Group}. 
\item<4-> \href{http://www.latex-community.org/}{\LaTeX{} Community}.
\end{enumerate}\label{bestHelp}
}

\frame{
The main site for packages is \href{http://www.tug.org/ctan.html}{CTAN: Comprehensive TeX Archive Network (maintained by the \TeX{} Users Group)}
}

\frame{
It would be hard to list all the on-line tutorials and reference documentation, but here are three:
\begin{enumerate}
\item<2-> \href{http://en.wikibooks.org/wiki/LaTeX}{Wikibooks' \LaTeX{} guide}
\item<3-> \href{http://www.dickimaw-books.com/latex/novices/}{Nicola Talbot, \textit{LaTeX for Complete Novices}}
\item<4-> \href{http://latex-project.org/help.html}{Help with LaTeX}
\end{enumerate}
}

\frame{\label{mittelbach}
Don't forget about \textbf{paper reference works}: 

These two items (also listed on \url{http://www.tug.org/books/}, among many others)

\begin{enumerate}
\item<2-> Frank Mittelbach, Goossens, \textit{et al.}, \textit{The \LaTeX{} Companion}. (Frank Mittelbach tells me a very good PDF version is supposed to appear on the \href{http://www.pearsonhighered.com/educator/product/LaTeX-Companion-The-2E/9780133387667.page}{Pearson website} shortly.)
\item<3-> Donald E. Knuth, \textit{The TeXbook}. (Source at \url{http://www.ctan.org/pkg/texbook}\uncover<4->{; note that it is licensed in such a way as to ``prevent distribution'' --- please respect the author's wishes. \uncover<5->{The \texttt{.tex} file is full of amusing commented remarks that do not appear in the PDF or printed volume.)}}
\end{enumerate}
}

\section{The Lore of \TeX{}}

\frame{
The Lore of \TeX{}\dots
}

\subsection{Version numbering}

\frame{\frametitle{Version numbering}
Version-numbering of \TeX{}: 

\uncover<2->{1}

\uncover<3->{2}

\uncover<4->{3\uncover<5->{.1\uncover<6->{4\uncover<7->{1\uncover<8->{5\uncover<9->{9\uncover<10->{2\uncover<11->{6\uncover<12->{. That's all so far.}}}}}}}}}
}

\subsection{The pronunciation of \TeX{} etc.}


\frame{\frametitle{The pronunciation of \TeX{} etc.}
Donald Knuth:

\begin{quote}English words like `technology' stem from a Greek root beginning with
the letters \(\tau\epsilon\chi\ldots\,\); and this same Greek word means \uline{art\/} as well as technology. Hence the name \TeX, which is an
uppercase form of \(\tau\epsilon\chi\).

\vskip9pt Insiders pronounce the \(\chi\) of \TeX\ as a Greek chi, not as an `x', so that
\TeX\ rhymes with the word blecchhh. It's the `ch' sound in Scottish words
like \uline{loch\/} or German words like \uline{ach\/}; it's a Spanish `j' and a
Russian `kh'. When you say it correctly to your computer, the terminal
may become slightly moist.
\end{quote}
\qquad\qquad\qquad\qquad\textit{The \TeX{}book}, v. 3.0 (1996), p. 1
}

\subsection{Separating content from formatting}

\frame{\frametitle{Separating content from formatting}
Advocates of \TeX{} often say that it enables you to separate content and formatting, allowing you to concentrate on the former.

\uncover<2->{\vskip12pt That would be true of any mark-up language, not just \TeX.}
}

\frame{\frametitle{Separating content from formatting}
I have two reasons for doubting the value of this argument.

\uncover<2->{\vskip6pt First, in reality, doing a good job with \TeX{} can take considerably longer than simply typing what you have to say into a standard word processor or text editor. A simpler mark-up language would be correspondingly more effective than \TeX{} at saving time and fuss spent on formatting. Good \TeX{} code is often quite complex.}

\uncover<3->{\vskip6pt I don't think this objection is answerable.}
}

\frame{\frametitle{Separating content from formatting, cont'd}
Second: cognitively, human brains do not distinguish form and content very well. Think of someone shouting at you --- do you really keep the message separate from how it is delivered? 

\uncover<2->{\vskip6pt I admit that that's an objection about reception of information. As a strategy for producing text, however, distinguishing the two may be useful.}
}

\frame{\frametitle{Trivia about \TeX{}}
It was originally invented to enable mathematical typesetting. Complexity followed.

\vskip12pt In the process of developing it, Knuth had to deal with the mathematics of both fonts and line-breaking, both highly non-trivial subjects.
}

\frame{\frametitle{About Donald Knuth}

Famous for developing several important algorithms, including one for the fast matching of strings.

\vskip12pt His books are permeated with humor. I am reminded of Bronowski's comment:

\begin{quote}
If you read Galileo's Dialogues and all those corny jokes and all that leg pulling, here is a man who is in love with his subject\dots
\end{quote}
\small{\qquad\qquad\qquad Jacob Bronowski (1908--1974), \textit{Magic, Science, and \vskip0pt\qquad\qquad\qquad  Civilization} (New York: Columbia University Press, \vskip0pt\qquad\qquad\qquad 1978), p. 36.}

}


\frame{
\begin{center}
\Huge END
\end{center}
}

\end{document}